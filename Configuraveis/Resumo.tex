% resumo em português
\setlength{\absparsep}{18pt} % ajusta o espaçamento dos parágrafos do resumo
\begin{resumo}
A reconstrução 3D de ambientes vêm sendo utilizada em diversos campos, como por exemplo: O mapeamento de ambientes para pesquisa arqueológica em áreas onde os cientistas não podem alcançar, ou comparar uma área antes e depois de uma catástrofe\cite{SLAMAP}, também em automação de robôs para que eles possam se locomover em um ambiente e agora, um dos mais recentes usos é a realidade virtual para jogos. Infelizmente, ainda há certas limitações, pois a câmera utilizada influencia imensamente no resultado, além do fato de que precisa estar devidamente calibrada. Normalmente para o mapeamento, são utilizadas várias câmeras, para obter diversos pontos de vista para um mesmo objeto ,semelhante à como os olhos humanos percebem o ambiente, e há técnicas e procedimentos para realizar o mapeamento com apenas uma câmera com a ajuda da \textit{Multiple View Geometry}. O objetivo deste trabalho é utilizar tais fundamentos para obter a reconstrução do ambiente, inicialmente de um \textit{smartphone}, resultando no aplicativo \textit{PhotoGuide}. Tal aplicativo reconhece \textit{keypoints} de um vídeo ou diretamente da câmera, mas por problemas de desempenho, foi necessário mover o processamento para máquinas mais poderosas, como um \textit{desktop} trocando a ferramenta de reconstrução para o \textit{LSD-SLAM}, que fornece um pacote completo de ferramentas para reconstruções 3D. 

 \textbf{Palavras-chave}: Reconstrução 3D. LSD-SLAM. Processamento de imagens. Mapeamento do Ambiente.

\end{resumo}