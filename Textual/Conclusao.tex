\chapter{Conclusão}
Como apresentado no capítulo 5, foi possível obter reconstruções parciais dos ambientes, externa e internamente. Essas reconstruções podem ser úteis em diversos aspectos, como servir de guia para pessoas com deficiência visual. Para isso precisaríamos do aplicativo de navegação, um possível trabalho futuro seria um aplicativo que recebesse o mapa 3D do ambiente, possivelmente processado pelo \textit{LSD-SLAM}, e permitisse a navegação do usuário. Uma outra forma seria a navegação virtual pelo ambiente, onde mudanças na orientação ou posição do \textit{smartphone} refletem no modelo, como é utilizado hoje na realidade aumentada.

Empresas poderiam se beneficiar dessa ferramenta para tornar seu ambiente de trabalho mais amigável para pessoas com deficiência. Com um maior refinamento das ferramentas e na extração do modelo, bem como a utilização de câmeras com melhor distância focal e maior ângulo de visão, seria possível realizar uma reconstrução com maior fidelidade. 

 Com o \textit{PhotoGuide}, foi possível obter \textit{keypoints} mas devido a tantas filtragens, em alguns pares de imagens o número de \textit{keypoints} era muito pouco, impossibilitando seu uso para processamento do \textit{dataset}. Futuramente mais testes com as filtragens podem ser feitos, para que haja o menor número possível de falsos-positivos mas ainda assim haja um bom número de casamentos. No momento grande parte dos celulares \textit{smartphones} comuns no mercado não possuem a capacidade computacional nem a qualidade de câmera e precisão de sensores boa o suficiente para a geração dessa nuvem de pontos pelo próprio \textit{smartphone} ao mesmo tempo que captura o vídeo. 
 
 Uma proposta interessante para solucionar isso em projetos futuros seria trabalhar com as estruturas de dados da saída de captura de pontos a fim de encontrar uma forma de fazer o máximo de computação que o \textit{smartphones} consegue. Então enviar a captura pela rede essa captura de forma otimizada, a fim de não consumir tanta banda quanto enviar uma imagem inteira, para poderem ser processadas em um servidor dedicado que possa executar a criação do mapa 3D usando \textit{GPUs} e que consiga enviar de volta o resultado para o \textit{smartphone}.
	
Outro projeto futuro seria criar um aplicativo \textit{mobile}, que utilize a nuvem de pontos resultante para permitir a locomoção de um usuário pelo ambiente mapeado, e se possível, utilizar a visão computacional juntamente com os dados obtidos dos sensores do \textit{smartphone} , para informar possíveis obstáculos ou paredes.
