\chapter{Conclusão}
Como apresentado no capítulo 5, foi possível obter reconstruções parciais dos ambientes, externa e internamente. Essas reconstruções podem ser úteis em diversos aspectos, como servir de guia para pessoas com deficiência visual ou se locomoverem em um ambiente desde que previamente mapeado, um exemplo disto seria uma empresa mais amigável para pessoas com deficiência, com um maior refinamento, bem como a utilização de câmeras mais sensíveis, pode ser possível realizar uma reconstrução com maior fidelidade. 

 Com o \textit{PhotoGuide}, foi possível obter \textit{keypoints} mas devido a tantas filtragens, em alguns pares o número de \textit{keypoints} era muito pouco, impossibilitando o uso. Futuramente mais testes com as filtragens devem ser feitos, para que haja o menor número possível de falso-positivos mas ainda assim haja um bom número de casamentos. No momento grande parte dos celulares \textit{smartphones} comuns no mercado não possuem a capacidade computacional nem a qualidade de câmera e precisão de sensores boa o suficiente para a geração dessa nuvem de pontos dentro do celular ao mesmo tempo que captura o vídeo. Uma proposta interessante para solucionar isso em projetos futuros seria trabalhar com as estruturas de dados da saída de captura de pontos a fim de encontrar uma forma de fazer o máximo de computação que o \textit{smartphones} consegue e enviar pela rede essa captura de forma otimizada, a fim de não consumir tanta banda quanto enviar uma imagem inteira, para poderem ser processadas em um servidor dedicado que possa executar a criação do mapa 3D usando \textit{GPUs} e que consiga enviar de volta o resultado para o \textit{smartphone}.
	
Outro projeto futuro seria criar um aplicativo \textit{mobile}, que utilize a nuvem de pontos resultante para permitir a locomoção de um usuário pelo ambiente mapeado, e se possível, utilizar a visão computacional juntamente com os dados obtidos dos sensores do \textit{smartphone} , para informar possíveis obstáculos ou paredes.
