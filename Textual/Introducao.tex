\chapter{Introdução}

Apresentaremos neste capítulo a motivação para este trabalho, problemas encontrados e outros trabalhos encontrados que também tentam reconstruir ambientes. A forma como o trabalho está estruturado e nossa proposta inicial será apresentada também.


\section{Motivação}

Alguns avanços tecnológicos que se possui hoje em dia graças à modelagem de ambientes que podem ser citados é a melhor investigação forense com o mapeamento das cenas de um crime [13], o carro autônomo em desenvolvimento pela Google ou a realidade virtual. Criando assim novos ramos de pesquisa, novos mercados à serem explorados.
	Um potencial uso, onde a vida de diversas pessoas seria mudada, é a utilização do mapeamento (seja ele on the fly ou realizado previamente) para guiar pessoas com deficiência visual num ambiente. Alguns trabalhos semelhantes foram realizados [referencia para trabalhos de wendel] mas diversas dificuldades foram encontradas. Uma delas é a questão de identificar a forma dos mais diversos objetos e também a topografia do ambiente. 



\section{Objetivos}

Os objetivo deste trabalho é realizar, mesmo que parcialmente, o mapeamento 3D do prédio do departamento de Ciências da Computação da UFS.

\subsection{Objetivos Específicos}

Para atingir o objetivo, foram utilizadas duas câmeras, uma câmera usb (modelo PSEye®) e a câmera nativa do smartphone Motorola Moto X Play 32GB, para obter imagens e vídeos do ambiente, o software LSD-SLAM [referencia para lsdslam] para modelagem e o software VLC para extrair os frames dos vídeos capturados a fim de criar os datasets. Ao longo do trabalho, foi criado um aplicativo para o sistema mobile Android, onde seria possível calibrar a câmera do dispositivo smartphone Android e também gravar os datasets necessários para o trabalho, na tentativa de dispensar a necessidade de um computador em todas as partes da obtenção e processamento do dataset.

\section{Abordagem utilizada}

\section{Estrutura do Documento}


Este trabalho está dividido da seguinte forma: No Capítulo 2 será apresentado o estado da arte sobre reconstrução 3D de ambientes; no Capítulo 3 a metodologia utilizada para obter os objetivos deste trabalho; no Capítulo 4 será abordados os resultados obtidos e por fim no Capítulo 5 nossa conclusão e possíveis trabalhos futuros.


Para facilitar a navegação e melhor entendimento, este documento está
estruturado em capítulos e seções, que são:
\begin{itemize}
\item {Capítulo 1 - Introdução}: xxx \cite{Yu:2004:ESG:1015090.1015207};
\item {Capítulo 2 - Estado da Arte}: xxx \cite{Cormen:2009};
\item {Capítulo 3 - Metodologia}: xxx \cite{Weicker:1984:DSS:358274.358283}
\item {Capítulo 4 - Resultados}: xxx \cite{IEEE_802_11:6178212};
\item {Capítulo 5 - Conclusão}: xxx \cite{Linux:402081};
\end{itemize}