\chapter{Introdução}

Apresentaremos neste capítulo a motivação para este trabalho, problemas encontrados e outros trabalhos que também tentam reconstruir ambientes. A forma como o trabalho está estruturado e a sua proposta inicial será apresentada também.

\section{Motivação}

Alguns avanços tecnológicos atualmente são graças à modelagem de ambientes, alguns deles que podem ser citados são: a melhor investigação forense com o mapeamento das cenas de um crime \cite{FIT3D}, o carro autônomo em desenvolvimento pela \textit{Google} \cite{GoogleX} e realidade virtual, criando assim novos ramos de pesquisa, novos mercados à serem explorados.

Um potencial uso, em que a vida de diversas pessoas seria mudada, é a utilização do mapeamento, seja ele \textit{on the fly} ou realizado previamente, para guiar pessoas com deficiência visual num ambiente. Alguns trabalhos semelhantes foram realizados mas diversas dificuldades foram encontradas. Uma delas é a questão de identificar a forma dos mais diversos objetos e também a topografia do ambiente. 


\section{Objetivos}

Os objetivo deste trabalho é realizar, mesmo que parcialmente, o mapeamento 3D de estruturas e ambientes, internos e externos, afim de utilizá-lo para outros fins potenciais. Tendo como estudo de caso, a reconstrução 3D do prédio do departamento de computação.

\subsection{Objetivos Específicos}

Para atingir o objetivo, foram utilizadas duas câmeras: uma câmera \textit{USB} do modelo PSEye® e a câmera nativa do smartphone \textit{Motorola Moto X Play 32GB} para obter imagens e vídeos do ambiente; o software \textit{LSD-SLAM} [12] para modelagem e o \textit{software VLC} para extrair os quadros dos vídeos capturados pelas câmeras a fim de criar os \textit{datasets}. Ao longo do trabalho, foi criado um aplicativo para o sistema mobile \textit{Android}, onde seria possível calibrar a câmera do dispositivo \textit{smartphone Android} e também gravar os \textit{datasets} necessários para o trabalho, na tentativa de dispensar a necessidade de um computador em todas as partes da obtenção e processamento do \textit{dataset}.


\section{Abordagem utilizada}

O trabalho possui as seguintes fases:
%considerar usar description lista no lugar
\begin{itemize}
\item{Estudo sobre o estado da arte: Foram pesquisados os diversos métodos disponíveis atualmente para mapeamento de ambientes, bem como seus pormenores. Também como  poderiam ser utilizados para o objetivo deste trabalho}
\item{Criação de um aplicativo para o sistema operacional \textit{Android}, onde fosse possível calibrar a câmera e obter um preprocessamento do \textit{dataset}, na forma de um banco de dados com quadros e seus \textit{keypoints}}
\item{Estudo e testes com as ferramentas encontradas: Várias ferramentas para reconstrução de ambientes foram encontradas, e nesta fase todas seriam testadas}
\item{Criação do \textit{dataset}: Vídeos e imagens do ambiente externo e interno do departamento}
\item{Processamento e avaliação do \textit{dataset}: Nesta fase foi realizado o processamento do \textit{dataset} com a ferramenta \textit{LSD-SLAM}, onde foi preciso realizar diversos testes para um melhor refinamento de suas configurações}
\end{itemize}


\section{Estrutura do Documento}


Este trabalho está dividido da seguinte forma: No Capítulo 2 será apresentado o estado da arte sobre reconstrução 3D de ambientes; no Capítulo 3 a metodologia utilizada para obter os objetivos deste trabalho; no Capítulo 4 serão abordados os resultados obtidos e por fim no Capítulo 5 a conclusão e possíveis trabalhos futuros.

